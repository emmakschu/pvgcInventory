\documentclass[titlepage]{article}

\usepackage[margin=1.25in]{geometry}
\usepackage{fancyhdr}


\title{Pleasant View Golf Course Maintenance Inventory App: \\ Project Requirements}
\author{Michael K Schumacher}
\date{\today}

\pagestyle{fancy}
\fancyhf{}
\fancyhead[R]{\leftmark}
\fancyhead[L]{PVGC Inventory App: Project Requirements}
\fancyfoot[C]{\thepage}

\renewcommand{\footrulewidth}{0.2pt}


\begin{document}

    \maketitle
    
    \vspace{20pt}
    
    \section{Background}
        During the week of 12-18 March, 2017, Pleasant View Golf Course (hereafter PVGC) Greens
        Superintendent Kevin Hurd informed me that Assistant Superintendent Matt Statz and 
        Mechanic Eric Meinholz desired a computer application to create and manage a database of
        their equipment and parts inventory. Currently the inventory is maintained with somewhat
        irregular counts and tracked on paper and/or Excel spreadsheets. A more convenient, more
        easily updated system is wanted.
        
        It is hoped that primary development and alpha-testing can be finished by mid-May, when
        the golf season begins to get especially busy. Since this is such a small-scale,
        personalized project, beta-testing will essentially be done by the enduser. Fixes,
         refactoring, and general improvements will be done on-the-go as deemed necessary.
        
    \section{Platform}
        The PVGC maintenance shop uses Windows 10 PCs. Development will be done primarily on Linux
        (Arch Linux) and Oracle VirtualBox Windows 10 Virtual Machines running on the Linux hosts.
        
        \subsection{Primary implementation language}
            To ensure maximum portability, the application will be written primarily in Java 8,
            with JavaFX used for building the GUI.
        
        \subsection{Database}
            Current plans are for a MariaDB implementation of the mysql server for the database.
            Connection to the SQL server will be achieved with the JDBC driver.
    
    \section{Data to catalog}
        NOTA BENE: The information in this section is subject to change based on client needs.
        
        \subsection{Equipment}
            For equipment items (e.g. greens mowers, sprayers, aerators, etc), the following
            parameters will need to be catalogued:
            
            \begin{itemize}
                \item
                Make
                
                \item
                Model
                
                \item
                Machine ID number (e.g. greens mower \#6)
                
                \item
                Model year
                
                \item
                Date purchased (estimated, if not known)
                
                \item
                Date info updated (\verb|CURRENT_TIMESTAMP|)
                
            \end{itemize}
        
        \subsection{Parts}
            Spare parts kept on hand will be tracked with the following parameters:
            
            \begin{itemize}
                \item
                Description (e.g. air filter)
                
                \item
                Manufacturer
                
                \item
                Manufacturers part number
                
                \item
                Manufacturer's price
                
                \item
                Alternative manufacturer 1
                
                \item
                Alternative manufacturer 1 part number
                
                \item
                Alternative manufacturer 1 price
                
                \item
                Equipment on which part is used
            \end{itemize}
            
    \section{GUI functionality}
        The GUI need not be especially flashy. Utility and ease of use should be the primary
        motivations.
        
        \subsection{Start page}
            On launch, the application should display a logo, and menu items from which the
            enduser will select which type of task he wishes to accomplish. Tasks should include:
            
            \begin{itemize}
                \item
                View inventory
                    \begin{itemize}
                        \item
                        View inventory by item category
                    \end{itemize}
                
                \item
                View individual item
                
                \item
                Edit individual item
                
                \item
                Edit multiple items
                    \begin{itemize}
                        \item
                        Edit items by category
                    \end{itemize}
                
                \item
                Search inventory
            \end{itemize}
            
            \subsection{View inventory}
                On viewing inventory, the GUI should display basic information about each item
                in a grid. Enduser should be able to click an item to view it individually for
                more detail. Grid should be setup similar to:
                
                \begin{tabular}{|l|l|l|l|} \hline
                    make & model & ID & year \\ \hline
                
                \end{tabular}

\end{document}